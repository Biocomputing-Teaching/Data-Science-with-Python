%%%%%%%%%%%%%%%%%%%%%%%%%%%%%%%%%%%%%%%%%
% Beamer Presentation
% LaTeX Template
% Version 1.0 (10/11/12) 
%
% This template has been downloaded from:
% http://www.LaTeXTemplates.com
%
% License:
% CC BY-NC-SA 3.0 (http://creativecommons.org/licenses/by-nc-sa/3.0/)
%
%%%%%%%%%%%%%%%%%%%%%%%%%%%%%%%%%%%%%%%%%

%----------------------------------------------------------------------------------------
%	PACKAGES AND THEMES
%----------------------------------------------------------------------------------------

\documentclass{beamer}

\mode<presentation> {
%\mode<handouts> {
%\mode<article> {


% The Beamer class comes with a number of default slide themes
% which change the colors and layouts of slides. Below this is a list
% of all the themes, uncomment each in turn to see what they look like.


%\usetheme{default}
%\usetheme{AnnArbor}
%\usetheme{Antibes}
%\usetheme{Bergen}
%\usetheme{Berkeley}
%\usetheme{Berlin}
%\usetheme{Boadilla}
\usetheme{CambridgeUS}
%\usetheme{Copenhagen}
%\usetheme{Darmstadt}
%\usetheme{Dresden}
%\usetheme{Frankfurt}
%\usetheme{Goettingen}
%\usetheme{Hannover}
%\usetheme{Ilmenau}
%\usetheme{JuanLesPins}
%\usetheme{Luebeck}
%\usetheme{Madrid}
%\usetheme{Malmoe}
%\usetheme{Marburg}
%\usetheme{Montpellier}
%\usetheme{PaloAlto}
%\usetheme{Pittsburgh}
%\usetheme{Rochester}
%\usetheme{Singapore}
%\usetheme{Szeged}
%\usetheme{Warsaw}

% As well as themes, the Beamer class has a number of color themes
% for any slide theme. Uncomment each of these in turn to see how it
% changes the colors of your current slide theme.

%\usecolortheme{albatross}
\usecolortheme{beaver}
%\usecolortheme{beetle}
%\usecolortheme{crane}
%\usecolortheme{dolphin}
%\usecolortheme{dove}
%\usecolortheme{fly}
%\usecolortheme{lily}
%\usecolortheme{orchid}
%\usecolortheme{rose}
%\usecolortheme{seagull}
%\usecolortheme{seahorse}
%\usecolortheme{whale}
%\usecolortheme{wolverine}

%\setbeamertemplate{footline} % To remove the footer line in all slides uncomment this line
%\setbeamertemplate{footline}[page number] % To replace the footer line in all slides with a simple slide count uncomment this line

%\setbeamertemplate{navigation symbols}{} % To remove the navigation symbols from the bottom of all slides uncomment this line
}

\usepackage{graphicx} % Allows including images
\graphicspath{{../figures}}
\usepackage{booktabs} % Allows the use of \toprule, \midrule and \bottomrule in tables
\usepackage{amsmath, amssymb, amsthm, gensymb,mathrsfs,bm,mlmath}%,eufrak}
\usepackage{hyperref}
\usepackage{tabularx}
\usepackage{longtable}
\usepackage{makecell}
\usepackage{multicol}
\usepackage{physics}

\newcommand{\uvec}[1]{\textbf{#1}}

\newcounter{excounter}
%\renewcommand{\thefpcounter}{\thechapter.\arabic{fpcounter}}
%\renewcommand{\thefpcounter}{\thesection.\arabic{fpcounter}}
\renewcommand{\theexcounter}{\arabic{excounter}}

\usepackage[lastexercise]{exercise}

\usepackage{fancyvrb}
\usepackage{xcolor}
\usepackage{listings}
\lstset{language=Python,
    basicstyle=\ttfamily,
    commentstyle=\color{red},
    keywordstyle=\color{blue},
    captionpos=b,
    backgroundcolor=\color{lightgray},
    showstringspaces=false
}

\renewcommand{\lstlistingname}{Code}

\definecolor{links}{HTML}{2A1B81}
\hypersetup{colorlinks,linkcolor=,urlcolor=links}
\setbeamertemplate{caption}[numbered]

\usepackage[linesnumbered,ruled,vlined,boxed]{algorithm2e}
%%% Coloring the comment as blue
\newcommand\mycommfont[1]{\footnotesize\ttfamily\textcolor{blue}{#1}}
\SetCommentSty{mycommfont}

\SetKwInput{KwInput}{Input}                % Set the Input
\SetKwInput{KwOutput}{Output}              % set the Output

\usepackage{fontspec}
\usepackage{unicode-math}
\setmathfont{Asana Math}
\graphicspath{{../figures}}

%----------------------------------------------------------------------------------------
%	 TITLE PAGE
%----------------------------------------------------------------------------------------

\title[Decision Trees]{Decision Trees} % The short title appears at the bottom of every slide, the full title is only on the title page

\author{Jordi Villà i Freixa} % Your name
\institute[FCTE] % Your institution as it will appear on the bottom of every slide, may be shorthand to save space
{
Universitat de Vic - Universitat Central de Catalunya \\
Study Abroad\\ % Your institution for the title page
\medskip
\textit{jordi.villa@uvic.cat} % Your email address
}
%\date{\today} % Date, can be changed to a custom date
\date{course 2023-2024}
\logo{\includegraphics[width=.1\textwidth]{FCTE}}
\begin{document}

\begin{frame}
\titlepage % Print the title page as the first slide
\end{frame}

\begin{frame}
\frametitle{Index} % Table of contents slide, comment this block out to remove it
\tableofcontents % Throughout your presentation, if you choose to use \section{} and \subsection{} commands, these will automatically be printed on this slide as an overview of your presentation
\end{frame}

%----------------------------------------------------------------------------------------
%	PRESENTATION SLIDES
%----------------------------------------------------------------------------------------
\begin{frame}
  \frametitle{Preliminary note}
  The material in these slides is strongly based on \cite{kroese2020}. When other materials are used, they are cited accordingly.

  Mathematical notation follows as good as it can a \href{https://ctan.math.utah.edu/ctan/tex-archive/macros/latex/contrib/mlmath/mlmath.pdf}{good practices proposal} from the Beijing Academy of Artificial Intelligence.
  \end{frame}

%\subsection{Subsection Example} % A subsection can be created just before a set of slides with a common theme to further break down your presentation into chunks

\begin{frame}{What to expect?}
  In this session we will discuss:
  \begin{itemize}
    \item Decision trees
    \item Random Forests
    \item Boosting
  \end{itemize}
\end{frame}

\section{Tree-based methods}


\begin{frame}{Tree-based methods}
\begin{itemize}
    \item Simple, intuitive and powerfiul for both regression and classification
    \item The method divides a feature space $X$ into smaller regions and fit a simple prediction function for each region.
    \begin{description}
        \item[Regression] eg, take the mean of the training responses associated with the training features that fall in the specific region
        \item[Classification] eg, take the majority vote among corresponding response variables.
    \end{description}
\end{itemize}
\end{frame}

\begin{frame}
    \begin{figure}
        \includegraphics[width=0.9\linewidth]{F81Kroese}
        \caption{Left: training data and a new feature. Right: a partition of the feature space\cite{kroese2020}}
        \label{Fig:F81Kroese}
    \end{figure}
\end{frame}

\begin{frame}
    \begin{itemize}
        \item In the example above, we cannot separate regions linearly, but we can create rectangles in the $\mathbb{R}^2$ space.
        \item The classifier $g$, thus takes a color "blue" or "red" according to where the new black dot goes.
        \item Both the classification procedure and the partitioning can be represented by a binary {\em decision tree}.
    \end{itemize}
    \end{frame}

\begin{frame}{Partitioning of feature space $X$}
    \begin{columns}
        \begin{column}{0.4\linewidth}
            \begin{figure}
                \includegraphics[width=0.7\linewidth]{F82Kroese}
                \label{Fig:F82Kroese}
            \end{figure}
        \end{column}
        \begin{column}{0.6\linewidth}
            The decision tree corresponding to Fig. \ref{Fig:F81Kroese}\cite{kroese2020}
            \begin{itemize}
                \item Each node $\nu$ corresponds to a region $\mathcal{R}_{\nu}$ of the feature space $X$. 
                \item The root node is the featre space $X$ itself.
                \item The final (undivided) leafs $w_1, w_2, \ldots$ (red/blue squares) form a partition of $X$, as they are disjoint and their union is $X$.
                \item Regional prediction functions $g^w$ are associated with each leaf.
            \end{itemize}
        \end{column}
    \end{columns}
\end{frame}

\begin{frame}{Decision}
    Let us take the input $\uvec{x}=(x_1,x_2)^T$:
    \begin{enumerate}
        \item we start at the root node, which contains a condition $x_2\leq 12.0$ that the input data satisifes.
        \item We then proceed to the left child which contains the condition $x_2 \leq -20.5$, which our data does not satisfies, so we go the next right leaf and so on.
    \end{enumerate}
    More generally, a binary tree $\mathbb{T}$ will partition the feature space into as many regions as leaf nodes, and the prediction function becomes:
    \[
        g(\uvec{x})=\sum_{w\in\mathcal{W}}g^w (\uvec{x})\mathbb{1}\{\uvec{x}\in \mathcal{R}_w\}   
    \]
\end{frame}

\section{Top down construction trees}

\begin{frame}
    \begin{figure}
        \includegraphics[width=0.9\linewidth]{A821Kroese}
    \end{figure}
\end{frame}

\begin{frame}
    \begin{figure}
        \includegraphics[width=0.7\linewidth]{F83Kroese}
        \caption{Entropy(normilized, divided by 2), Gini and missclassification impurities in binary classification.\cite{kroese2020}}
    \end{figure}
\end{frame}

\begin{frame}
    \begin{figure}
        \includegraphics[width=0.7\linewidth]{F84Kroese}
        \caption{Ten-fold crossvalidation loss fnction as a function of the maximal tree depth in a classification problem.\cite{kroese2020}}
    \end{figure}
\end{frame}

\section{Additional considerations}

\begin{frame}
    \begin{figure}
        \includegraphics[width=0.9\linewidth]{F85Kroese}
        \caption{The two groups are separated, in decision tree, with a collection of rectangles, leading to an unnecessary classification procedure some times.\cite{kroese2020}}
    \end{figure}
\end{frame}

\section{Controlling the tree shape}

\begin{frame}
    \begin{figure}
        \includegraphics[width=0.7\linewidth]{F86Kroese}
        \caption{Cross-validation and training loss as a function of the tree depth dfor a binary classification problem.\cite{kroese2020}}
    \end{figure}
\end{frame}

\begin{frame}
    \begin{figure}
        \includegraphics[width=0.7\linewidth]{F87Kroese}
        \caption{Different ancestors for different nodes.\cite{kroese2020}}
    \end{figure}
\end{frame}

\begin{frame}
    \begin{figure}
        \includegraphics[width=0.9\linewidth]{F88Kroese}
        \caption{Pruning branches of the tree.\cite{kroese2020}}
    \end{figure}
\end{frame}

\section{Boostrapping aggregation}

\begin{frame}
    \begin{figure}
        \includegraphics[width=0.9\linewidth]{A851Kroese}
    \end{figure}
\end{frame}

\begin{frame}
    \begin{figure}
        \includegraphics[width=0.9\linewidth]{A852Kroese}
    \end{figure}
\end{frame}

\section{Random Forests}

\begin{frame}
    \begin{figure}
        \includegraphics[width=0.9\linewidth]{A861Kroese}
    \end{figure}
\end{frame}

\begin{frame}
    \begin{figure}
        \includegraphics[width=0.7\linewidth]{F89Kroese}
        \caption{Importance measure for a 15-feature data set with only informative features $x_1,x_2,x_3,x_4,x_5$\cite{kroese2020}.}
    \end{figure}
\end{frame}

\section{Boosting}

\begin{frame}
    \begin{figure}
        \includegraphics[width=0.9\linewidth]{A871Kroese}
    \end{figure}
\end{frame}

\begin{frame}
    \begin{figure}
        \includegraphics[width=0.9\linewidth]{F810Kroese}
        \caption{Fitted boosting regression model with two different values of $\gamma$. \cite{kroese2020}}
    \end{figure}
\end{frame}

\begin{frame}
    \begin{figure}
        \includegraphics[width=0.7\linewidth]{F811Kroese}
        \caption{Exponential and zero-one training loss as a function of the number of boosting rounds $B$ for a binary classification problem. \cite{kroese2020}}
    \end{figure}
\end{frame}

\section{Bibliography}
\bibliographystyle{unsrt}
\bibliography{DataSciencewithPython}
\end{document}
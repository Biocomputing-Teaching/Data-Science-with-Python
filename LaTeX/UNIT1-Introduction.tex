%%%%%%%%%%%%%%%%%%%%%%%%%%%%%%%%%%%%%%%%%
% Beamer Presentation
% LaTeX Template
% Version 1.0 (10/11/12) 
%
% This template has been downloaded from:
% http://www.LaTeXTemplates.com
%
% License:
% CC BY-NC-SA 3.0 (http://creativecommons.org/licenses/by-nc-sa/3.0/)
%
%%%%%%%%%%%%%%%%%%%%%%%%%%%%%%%%%%%%%%%%%

%----------------------------------------------------------------------------------------
%	PACKAGES AND THEMES
%----------------------------------------------------------------------------------------

\documentclass{beamer}

\mode<presentation> {
%\mode<handouts> {
%\mode<article> {


% The Beamer class comes with a number of default slide themes
% which change the colors and layouts of slides. Below this is a list
% of all the themes, uncomment each in turn to see what they look like.


%\usetheme{default}
%\usetheme{AnnArbor}
%\usetheme{Antibes}
%\usetheme{Bergen}
%\usetheme{Berkeley}
%\usetheme{Berlin}
%\usetheme{Boadilla}
\usetheme{CambridgeUS}
%\usetheme{Copenhagen}
%\usetheme{Darmstadt}
%\usetheme{Dresden}
%\usetheme{Frankfurt}
%\usetheme{Goettingen}
%\usetheme{Hannover}
%\usetheme{Ilmenau}
%\usetheme{JuanLesPins}
%\usetheme{Luebeck}
%\usetheme{Madrid}
%\usetheme{Malmoe}
%\usetheme{Marburg}
%\usetheme{Montpellier}
%\usetheme{PaloAlto}
%\usetheme{Pittsburgh}
%\usetheme{Rochester}
%\usetheme{Singapore}
%\usetheme{Szeged}
%\usetheme{Warsaw}

% As well as themes, the Beamer class has a number of color themes
% for any slide theme. Uncomment each of these in turn to see how it
% changes the colors of your current slide theme.

%\usecolortheme{albatross}
\usecolortheme{beaver}
%\usecolortheme{beetle}
%\usecolortheme{crane}
%\usecolortheme{dolphin}
%\usecolortheme{dove}
%\usecolortheme{fly}
%\usecolortheme{lily}
%\usecolortheme{orchid}
%\usecolortheme{rose}
%\usecolortheme{seagull}
%\usecolortheme{seahorse}
%\usecolortheme{whale}
%\usecolortheme{wolverine}

%\setbeamertemplate{footline} % To remove the footer line in all slides uncomment this line
%\setbeamertemplate{footline}[page number] % To replace the footer line in all slides with a simple slide count uncomment this line

%\setbeamertemplate{navigation symbols}{} % To remove the navigation symbols from the bottom of all slides uncomment this line
}

\usepackage{graphicx} % Allows including images
\graphicspath{{../figures}}
\usepackage{booktabs} % Allows the use of \toprule, \midrule and \bottomrule in tables
\usepackage{amsmath, amssymb, amsthm, gensymb,mathrsfs,bm,mlmath}%,eufrak}
\usepackage{hyperref}
\usepackage{tabularx}
\usepackage{longtable}
\usepackage{makecell}
\usepackage{multicol}
\usepackage{physics}

\newcommand{\uvec}[1]{\textbf{#1}}

\newcounter{excounter}
%\renewcommand{\thefpcounter}{\thechapter.\arabic{fpcounter}}
%\renewcommand{\thefpcounter}{\thesection.\arabic{fpcounter}}
\renewcommand{\theexcounter}{\arabic{excounter}}

\usepackage[lastexercise]{exercise}

\usepackage{fancyvrb}
\usepackage{xcolor}
\usepackage{listings}
\lstset{language=Python,
    basicstyle=\ttfamily,
    commentstyle=\color{red},
    keywordstyle=\color{blue},
    captionpos=b,
    backgroundcolor=\color{lightgray},
    showstringspaces=false
}

\renewcommand{\lstlistingname}{Code}

\definecolor{links}{HTML}{2A1B81}
\hypersetup{colorlinks,linkcolor=,urlcolor=links}
\setbeamertemplate{caption}[numbered]

\usepackage[linesnumbered,ruled,vlined,boxed]{algorithm2e}
%%% Coloring the comment as blue
\newcommand\mycommfont[1]{\footnotesize\ttfamily\textcolor{blue}{#1}}
\SetCommentSty{mycommfont}

\SetKwInput{KwInput}{Input}                % Set the Input
\SetKwInput{KwOutput}{Output}              % set the Output

\usepackage{fontspec}
\usepackage{unicode-math}
\setmathfont{Asana Math}
\graphicspath{{../figures}}

%----------------------------------------------------------------------------------------
%	 TITLE PAGE
%----------------------------------------------------------------------------------------

\title[Introduction]{Improting, summarizing and visualizing data} % The short title appears at the bottom of every slide, the full title is only on the title page

\author{Jordi Villà i Freixa} % Your name
\institute[FCTE] % Your institution as it will appear on the bottom of every slide, may be shorthand to save space
{
Universitat de Vic - Universitat Central de Catalunya \\
Study Abroad\\ % Your institution for the title page
\medskip
\textit{jordi.villa@uvic.cat} % Your email address
}
%\date{\today} % Date, can be changed to a custom date
\date{ccourse 2023-2024}
\logo{\includegraphics[width=.1\textwidth]{FCTE}}
\begin{document}

\begin{frame}
\titlepage % Print the title page as the first slide
\end{frame}

\begin{frame}
\frametitle{Índex} % Table of contents slide, comment this block out to remove it
\tableofcontents % Throughout your presentation, if you choose to use \section{} and \subsection{} commands, these will automatically be printed on this slide as an overview of your presentation
\end{frame}

%----------------------------------------------------------------------------------------
%	PRESENTATION SLIDES
%----------------------------------------------------------------------------------------

\begin{frame}
  \frametitle{Introduction to the course}
  The material in these slides is strongly based on \cite{kroese2020}. When other materials are going to be used, they will be cited accordingly.
  \end{frame}

%------------------------------------------------
\section{Dealing with data} % Sections can be created in order to organize your presentation into discrete blocks, all sections and subsections are automatically printed in the table of contents as an overview of the talk
%------------------------------------------------

%\subsection{Subsection Example} % A subsection can be created just before a set of slides with a common theme to further break down your presentation into chunks


\begin{frame}
\frametitle{How data is stored?}
\begin{itemize}
  \item Data canb be thought of as being the result of some random experiment.
  \item We are interested in analysisng such data.
  \item The format is typically a set of variables or {\em features} as {\bf columns} while the different items are given as {\bf rows}.
  \item Typically the first columns represents a unique identifier or index.
  \item Some columns refer to the experimental settings and others are real variables.
  \item Many times variables and experimental designs are stored in two different files. The we call the experimental desgins file as the {\bf metadata} file, describing the details of the different experiments (or columns).  
\end{itemize}
\end{frame}

%------------------------------------------------

\begin{frame}
\frametitle{Trainign datasets}
There exist several datasets repositories that one can use to test the methods thyat are being developed. Some of them are woing to be used in this course:
\begin{itemize}
\item Machine Learning Repository at University of California (\url{https://archive.ics.uci.edu})
\item Vincent Arel-Bundock repository (\url{https://vincentarelbundock.github.io/Rdatasets/})
\end{itemize}
\end{frame}

%------------------------------------------------
\begin{frame}
\frametitle{Nombres discrets vs nombres continus}
\begin{itemize}
\item $\mathbb{Z}$ és un conjunt de nombres discrets: donat un enter, sempre hi ha un enter consecutiu. Exemple: el codi binari, ${0,1}$.
\item Un conjunt de números es diu que és continu si poden prendre qualsevol valor en un interval finit o infinit. Exemples: $]3,5]$, $(-\inf,0)$. El món no és discret, sinó {\bf mesurable}! Per tant, no podem parlar de dos nombres reals consecutius.
\end{itemize}
\begin{exampleblock}{La recta real}
  Els nombres reals es representen damunt una recta, la recta real.
\end{exampleblock}
\end{frame}

%------------------------------------------------
\begin{frame}
\frametitle{Propietat de la recta real}
\begin{itemize}
\item Necessitem una referència per anomenar els punts de la recta: fixarem un orígen (el $0$) i una escala ($1$)
\item La recta està ordenada
\item És infinita
\item els intèrvals i semirectes són parts de la recta:
  \begin{eqnarray}
    [a,b] &=& \{x \in \mathbb{R}: a \leq x \leq b\} \nonumber \\
    (a,b) &=& \{x \in \mathbb{R}: a < x < b\} \nonumber \\
    \left[a,\infty\right] &=& \{x \in \mathbb{R}: a \leq x\} \nonumber \\
    (-\infty, b) &=& \{x\in \mathbb{R}:x<b\}
  \end{eqnarray}
\item La distància entre nombres reals és $d(a,b) = |b-a|$.
\end{itemize}
\end{frame}

\section{Espais 2D i 3D}
%------------------------------------------------
\begin{frame}
\frametitle{El pla 2D}
\begin{itemize}
\item Fem el salt a dues dimensions. Ncessitarem referenciar els punts en un pla.
\item René Descartes (1590-1650) va posar les bases matemàtiques per a poder fer-ho: el producte cartesià $\mathbb{R} \times \mathbb{R}$, consistent en el conjunt de parells (ordenats) de nombres reals:
\begin{equation}
  \mathbb{R}^2= \mathbb{R} \times \mathbb{R} = \left\{ (x,y): x \in \mathbb{R}, y \in \mathbb{R} \right\}
\end{equation}
\end{itemize}
\end{frame}
%------------------------------------------------
\begin{frame}
\frametitle{El pla 2D}
\begin{itemize}
\item Com fèiem amb la recta, usem una referència per identificar els punts del pla: fixem un origen $(0,0)$ i dos punts que ens donguin l'escala horitzontal i vertical: $\{(1,0),(0,1)\}$.
\item Els eixos d'abcisses i ordenades venen determinats per les rectes $\{(x,y)\in \mathbb{R}^2: y=0\}$ i $\{(x,y)\in \mathbb{R}^2: x=0\}$, respectivament.
\end{itemize}
\begin{figure}
\includegraphics[width=0.6\linewidth]{FCTE}
\end{figure}
\end{frame}
%------------------------------------------------
\begin{frame}
\frametitle{El pla 2D}
\begin{itemize}
  \item Podem definir els quatre quadrants del pla com:
\begin{eqnarray*}
  Q1 = \{(x,y) \in \mathbb{R}^2 : x \geq 0, y \geq 0 \} \\
  Q2 = \{(x,y) \in \mathbb{R}^2 : x \leq 0, y \geq 0 \} \\
  Q3 = \{(x,y) \in \mathbb{R}^2 : x \leq 0, y \leq 0 \} \\
  Q4 = \{(x,y) \in \mathbb{R}^2 : x \geq 0, y \leq 0 \}
\end{eqnarray*}
  \item La distància euclídea:
\begin{equation}
  d\left( (x_1,y_1),(x_2,y_2) \right) = \sqrt{(x_1-x_2)^2+(y_1-y_2)^2}
\end{equation}
\end{itemize}

\end{frame}

%------------------------------------------------
\begin{frame}
\frametitle{L'Espai 3D}
\begin{itemize}
  \item El nostre interès és l'espai tridimensional, que ens apropa a la realitat (escultura, holografia, impressió-3D, animació 3D...).
  \item L'espai 3D és el conjunt de tríos de nombres reals:
  \begin{equation}
    \mathbb{R}^3 = \mathbb{R} \times \mathbb{R} \times \mathbb{R} = \{(x,y,z): x \in \mathbb{R}, y \in \mathbb{R}, z \in \mathbb{R} \}
  \end{equation}
  \item Triem un origen $(0,0,0)$ i tres punts més que ens donen les tres escales, $(1,0,0)$, $(0,1,0)$ i $(0,0,1)$
  \item podem definir eixos (per exemple l'eix $Z$ es defineix com $\{(x,y,z)\in \mathbb{R}^3:x=0, y=0\}$) o plans (l'Eix $XY$ vidra definit per $\{(x,y,z)\in \mathbb{R}^3:z=0\}$).
  \item La distància euclídea:
\begin{equation}
  d\left( (x_1,y_1,z_1),(x_2,y_2,z_2) \right) = \sqrt{(x_1-x_2)^2+(y_1-y_2)^2+(z_1-z_2)^2}
\end{equation}
\end{itemize}
\end{frame}

\section{Desplaçament i vectors}
%------------------------------------------------
\begin{frame}
\frametitle{Desplaçament i vectors}

El nostre interès és descriure moviments, ja siguin en la recta, en el pla o en l'espai 3D.

\begin{exampleblock}{Desplaçament a la recta real}
Com descriuries el desplaçament d'un punt des de la posició $x=2$ a la posició $x=4$? i el desplaçament invers?
\end{exampleblock}

\begin{exampleblock}{Desplaçament al Pla 2D}
Com descriuries el desplaçament rectilini d'un punt des de la posició $(2,1)$ a la $(-3,2)$. I el desplaçament contrari? D
óna dos punts inicial i final entre els quels hi hauria el mateix desplaçament.
\end{exampleblock}

\begin{block}{Desplaçament a l'espai 3D}
Pots posar un exemple similar en l'espai 3D?
\end{block}
\end{frame}

%------------------------------------------------
\begin{frame}
%\frametitle{Discret vs contínu}

En general:

\begin{exampleblock}{Desplaçament a la recta real}
El vector desplaçament entre la posició inicial $A$ i la final $B$ es defineix com $\vec{AB}= B-A$
\end{exampleblock}

\begin{exampleblock}{Desplaçament al Pla 2D}
El vector desplaçament entre la posició inicial $A=(x_1,y_1)$ i la final $B(x_2,y_2)$ es defineix com $\vec{AB}=(x_2-x_1,y_2-y_1)$
\end{exampleblock}

\begin{exampleblock}{Desplaçament a l'espai 3D}
El vector desplaçament entre la posició inicial $A=(x_1,y_1,z_1)$ i la final $B=(x_2,y_2,z_2)$ es defineix com $\vec{AB}=(x_2-x_1,y_2-y_1,z_2-z_1)$
\end{exampleblock}
Cada escalar del vector s'anomena component
\end{frame}

%------------------------------------------------
\begin{frame}
%\frametitle{Discret vs contínu}

\begin{exercise}{ex:pla2}{}
Donats els punts $A(1,1)$, $B=(0,-1)$ i el vector $\vec{u}=(2,4)$
\begin{enumerate}
  \item Calcula els vectors que van des de l'origen de coordenades cap a cadascun dels punts $A$ i $B$ (pregunta trampa...).
  \item Calcula i dibuixa $\overrightarrow{AB}$.
  \item Calcula i dibuixa $\overrightarrow{BA}$.
  \item Si $\vec{u}=\overrightarrow{AC}$, quina posició és $C$?
  \item Si $\vec{u}=\overrightarrow{CB}$, quina posició és $C$?
  \item Dóna una altres dos punts $A$ i $B$ que tinguin el mateix vector desplaçament $\overrightarrow{AB}$.
\end{enumerate}
\end{exercise}

\end{frame}
%------------------------------------------------
\begin{frame}
%\frametitle{Discret vs contínu}
Els desplaçaments, doncs, es representen amb vectors, quines principals característiques són:
\begin{itemize}
  \item Tenen una direcció.
  \item Tenen un sentit.
  \item Tenen una intensitat, anomenada {\bf norma} o {\bf mòdul} del vector, que és la distància entre la posició inicial i la final:
  \begin{equation}
  ||\overrightarrow{AB}|| = d(A,B)=\sqrt{(x_1-x_2)^2+(y_1-y_2)^2}
  \end{equation}
  si $A=(x_1,y_1)$ i $B=(x_2,y_2)$ són dos punts en el pla.
\end{itemize}
\begin{exercise}{}{}
  Calcula la norma del vector d'$\mathbb{R}^5$ $(1,-2,-3,0,2)$.
\end{exercise}


\end{frame}
%------------------------------------------------
\begin{frame}
%\frametitle{Discret vs contínu}

\begin{exercise}{ex:pla3}{}
Donats el punt $A(1,1)$ i els vectors $\vec{u}=(2,4)$, $\vec{v}=(0,-3)$:
\begin{enumerate}
  \item Aplica al punt $A$ el desplaçament $\vec{u}$, i al nou punt trobat el desplaçament $\vec{v}$. Quin és el vector desplaçament des d'$A$ a la posició final?
  \item Repeteix l'exercise canviant l'ordre dels desplaçaments.
  \item Aplica al punt $A$ el desplaçament $\vec{u}$, i al nou punt trobat el desplaçament $\vec{u}$ novament. Quin és el vector desplaçament des d'$A$ a la posició final?
\end{enumerate}
\end{exercise}

\end{frame}
%------------------------------------------------
\begin{frame}
%\frametitle{Discret vs contínu}

\begin{figure}
\includegraphics[width=0.8\linewidth]{FCTE}
\end{figure}
La suma de vectors és commutativa.

A partir d'ara considerarem els vectors com a entitats pròpies, més enllà del concepte de desplaçament.

\end{frame}

\section{Espais vectorials}

%------------------------------------------------
\begin{frame}
\frametitle{Espais vectorials}

Considerem el conjunt de tots els possibles vectors al pla amb origen a $(0,0)$. En realitat, es tracta de tots els parells ordenats de nombres reals: el conjunt $\mathbb{R}^2$.

El conjunt de tots els vectors de $\mathbb{R}^2$, amb les operacions definides per a qualsevol parell de vectors $\vec{u}=(u_1,u_2)$ i $\vec{v}=(v_1,v_2)$, i qualsevol nombre real $\lambda$ de la següent manera
\begin{enumerate}
\item {\bf Suma} (interna) $\vec{u}+\vec{v}=(u_1+v_1,u_2+v_2)$, i
\item {\bf Producte per escalar} (externa) $\lambda \cdot \vec{u} = (\lambda u_1, \lambda u_2)$
\end{enumerate}
satisfà les propietats següents $\forall \lambda, \gamma \in \mathbb{R}  , \forall \vec{u}, \vec{v},\vec{w}\in \mathbb{R}^2$:

\end{frame}
%------------------------------------------------
\begin{frame}
  \begin{enumerate}
    \item Propietat commutativa de l'operació interna: $\vec{u}+\vec{v}=\vec{v}+\vec{u}$
    \item Propietat associativa de l'operació interna: $(\vec{u}+\vec{v})+\vec{w}= \vec{u}+(\vec{v}+\vec{w})$
    \item Existeix un element neutre $\vec{e}$ tal que $\vec{u}+\vec{e}=\vec{u}$
    \item Existeix un element oposat $-\vec(u)$ per a cada vector $\vec{u}$ tal que $-\vec{u}+\vec{u}=\vec{e}$
    \item $\lambda \cdot (\gamma \cdot \vec{u})=(\lambda \gamma) \cdot \vec{u}$
    \item $(\lambda + \gamma) \cdot \vec{u} = \lambda \cdot \vec{u} + \gamma \cdot \vec{u}$
    \item $\lambda \cdot (\vec{u}+\vec{v}) = \lambda \cdot \vec{u} + \lambda \cdot \vec{v}$
    \item Existeix un element neutre $e \in \mathbb{R}$ tal que $e \cdot \vec{u} = \vec{u}$

  \end{enumerate}
Qui són $\vec{e}$ (a $\mathbb{R}^2$) i $e$? Passa el mateix a $\mathbb{R}$ o $\mathbb{R}^3$? I a $\mathbb{R}^n$?
\end{frame}

%------------------------------------------------
\begin{frame}
  \begin{definicio}
    Un conjunt d'elements (anomenats vectors) amb dues operacions externa i interna que satisfan les 8 propietats anteriors s'anomena {\bf espai vectorial} (sobre el cos escalar dels nombres reals)
  \end{definicio}
  $\mathbb{R}$ , $\mathbb{R}^2$ i $\mathbb{R}^3$ són espais vectorials sobre $\mathbb{R}$.

\begin{exercise}{}{}
  Comprova que el conjunt dels polinomis de grau 2 és un espai vectorial sobre $\mathbb{R}$.
\end{exercise}

\end{frame}


%------------------------------------------------
\begin{frame}
  \begin{definicio}
    Un vector $\vec{u}$ és una {\bf combinació lineal} dels vectors $\overrightarrow{v_1},\overrightarrow{v_2}, \ldots , \overrightarrow{v_n}$ si existeixen nombres reals $\lambda_1, \lambda_2, \ldots, \lambda_n$ que satisfan:
    \[\vec{u} = \lambda_1 \overrightarrow{v_1} + \lambda_2 \overrightarrow{v_2} + \cdots \lambda_n \overrightarrow{v_n}\]
  \end{definicio}

\begin{exercise}{}{}
  a) Troba els valors dels coeficients si $\vec{u}=(2,1)$ i $\{\overrightarrow{v_1},\overrightarrow{v_2}\} = \{\overrightarrow{e_1},\overrightarrow{e_2}\} = \{ (1,0),(0,1) \}$. b) i si $\{\overrightarrow{v_1},\overrightarrow{v_2}\} = \{ (2,1),(-2,1) \}$?
\end{exercise}

  \begin{exercise}{}{}
    Agafant els vectors de l'exercise anterior:
    \begin{enumerate}
      \item Pots escriure $\overrightarrow{e_2}$ en funció de $\{\overrightarrow{v_1},\overrightarrow{v_2}\}$?
      \item Pots escriure $\overrightarrow{e_1}$ en funció de $\{\overrightarrow{v_1},\overrightarrow{v_2}\}$?
      \item Pots escriure $\overrightarrow{v_2}$ en funció de $\{\overrightarrow{e_1}\}$?
      \item Com són els vectors que es poden escriure com a combinació lineal de $\overrightarrow{e_1}$?
    \end{enumerate}
  \end{exercise}
\end{frame}
%------------------------------------------------
\begin{frame}
  \frametitle{Dependència lineal}

    Els vectors $\overrightarrow{v_1},\overrightarrow{v_2}, \ldots, \overrightarrow{v_n}$ són {\bf linealment dependents} si qualsevol d'ells es pot escriure com a combinació lineal de la resta. En cas contrari els anomenem {\bf linealment independents}

  \begin{exercise}{}
    Amb els mateixos vectors anteriors:
    \begin{enumerate}
      \item Són $\overrightarrow{e_2}$, $\overrightarrow{v_1}$ i $\overrightarrow{v_2}$ linealment independents?
      \item Són $\overrightarrow{e_1}$, $\overrightarrow{v_1}$ i $\overrightarrow{v_2}$ linealment independents?
      \item Com són els vectors linealment dependents amb $\overrightarrow{e_1}$?
      \item Són $\overrightarrow{e_1}$ i $\overrightarrow{v_2}$ linealment independents?
      \item Són $\overrightarrow{e_1}$ i $\overrightarrow{e_2}$ linealment independents?
      \item Són $\overrightarrow{v_1}$ i $\overrightarrow{v_2}$ linealment independents?
    \end{enumerate}
  \end{exercise}
\end{frame}
%------------------------------------------------
\begin{frame}
  \frametitle{Vectors generadors}

    Els vectors $\overrightarrow{v_1},\overrightarrow{v_2}, \ldots, \overrightarrow{v_n}$ són generadors de l'espai vectorial al qual pertanyen quan qualsevol vector de l'espai es pot posar com a combinació lineal de $\overrightarrow{v_1},\overrightarrow{v_2}, \ldots, \overrightarrow{v_n}$.

    Per indicar que $\overrightarrow{v_1},\overrightarrow{v_2}, \ldots, \overrightarrow{v_n}$ generen l'espai vectoria $E$, escrivim: $E=\left<\overrightarrow{v_1},\overrightarrow{v_2}, \ldots, \overrightarrow{v_n}\right>$.


  \begin{exercise}{}
    Amb els mateixos vectors anteriors:
    \begin{enumerate}
      \item Comprova que $\overrightarrow{e_1}$ i $\overrightarrow{e_2}$ són generadors de $\mathbb{R}^2$.
      \item Comprova que $\overrightarrow{e_1}$, $\overrightarrow{v_1}$ i $\overrightarrow{v_2}$ són generadors de $\mathbb{R}^2$.
      \item Ho són $\overrightarrow{v_1}$ i $\overrightarrow{v_2}$?
      \item Dóna exemples de conjunts de vectors d'$\mathbb{R}^2$ que generin altres vectors del mateix espai vectorial amb la forma $\{(\alpha,0):\alpha \in  \mathbb{R} \}$.
      \item Comprova que el conjunt de vectors $\{(1,0,0),(0,1,0),(0,0,1)\}$ genera $\mathbb{R}^3$.
    \end{enumerate}
  \end{exercise}
\end{frame}
%------------------------------------------------
\begin{frame}
  \frametitle{Base d'un espai vectorial}

    Els vectors $\overrightarrow{v_1},\overrightarrow{v_2}, \ldots, \overrightarrow{v_n}$ són una base de l'espai vectorial al qual pertanyen quan:
    \begin{enumerate}
      \item són generados de l'espai, i
      \item són linealment independents.
    \end{enumerate}

  \begin{exercise}{}
    \begin{enumerate}
      \item Comprova que $\overrightarrow{e_1}$ i $\overrightarrow{e_2}$ és una base de $\mathbb{R}^2$.
      \item Perquè $\overrightarrow{e_1}$, $\overrightarrow{v_1}$ i $\overrightarrow{v_2}$ no són una base de $\mathbb{R}^2$?
      \item Formen una base de $\mathbb{R}^2$ els vectors $\overrightarrow{v_1}$ i $\overrightarrow{v_2}$?
      \item Quants vectors com a molt formen una base de $\mathbb{R}^2$?
      \item I d'$\mathbb{R}^3$?
    \end{enumerate}
  \end{exercise}
\end{frame}
%------------------------------------------------%------------------------------------------------
\begin{frame}
  \frametitle{Dimensió d'un espai vectorial}

    En un espai vectorial hi ha infinites bases, però totes tenen el mateix nombre de vectors. la {\bf dimensió} d'un espai vectorial és el nombre de vectors que es troben en una base.

    \begin{itemize}
      \item $\mathbb{R}$ té dimensió 1.
      \item $\mathbb{R}^2$ té dimensió 2.
      \item $\mathbb{R}^n$ té dimensió $n$.
    \end{itemize}

    Per exemple, la {\bf base canònica} de $\mathbb{R}^2$ és $\{\overrightarrow{e_1},\overrightarrow{e_2}\}=\{(1,0),(0,1)\}$.

\end{frame}
%------------------------------------------------
\begin{frame}
  \frametitle{Representació d'un vector en una base}

    Tot vector d'un espai vectorial es pot expressar com a combinació lineal dels vectors de qualsevol base. Les components d'un vector respecte a una base són els nombres reals que multipliquen cada vector de la base.

    Per exemple (veure exercise \ref{ex:base}), si $\overrightarrow{u}=(2,1)$ la seva representació en dues bases diferents $C=\{\overrightarrow{e_1},\overrightarrow{e_2}\}=\{(1,0),(0,1)\}$, i $B=\{\overrightarrow{v_1},\overrightarrow{v_2}\}=\{(-1,1)_C,(3,3)_C\}$ seria:

    \begin{eqnarray*}
      \vec{u}&=&2\cdot \overrightarrow{e_1}+ 1 \cdot \overrightarrow{e_2}=(2,1)_C=2\uvec{i}+\uvec{j} \\
      \vec{u}&=&\frac{-1}{2}\cdot \overrightarrow{v_1}+ \frac{1}{2} \cdot \overrightarrow{v_2}=(\frac{-1}{2},\frac{1}{2})_B
    \end{eqnarray*}

    Observació: sovint anomenem $\uvec{i},\uvec{j},\uvec{k}$ els vectors $\overrightarrow{e_1},\overrightarrow{e_2},\overrightarrow{e_2}$ a $\mathbb{R}^3$. En general, és pràctic escriure $\vec{u}$ com a $\uvec{u}$ per simplicitat.
\end{frame}
%------------------------------------------------
\begin{frame}
  \frametitle{Supespai vectorial}
  \begin{definicio}
    Un subconjunt de vectors d'un espai vectorial $F \subset E$ forma un subespai vectorial si i només si:
    \begin{enumerate}
      \item $\vec{u}+\vec{v} \in F$, sempre que $\vec{u},\vec{v} \in F$, i
      \item $\lambda \cdot \vec{u} \in F$, sempre que $\vec{u} \in F$ i $\lambda \in \mathbb{R}$.
    \end{enumerate}
  \end{definicio}
  Els subespais vectorials son també espais vectorials i, com a tals, tenen bases i dimensió.

A $\mathbb{R}^2$ hi ha infinits subespais vectorials de dimensió 1 (infinites rectes que passen per l'origen en el pla).
\end{frame}
%------------------------------------------------
\begin{frame}
  \begin{exercise}{}{}
    A $\mathbb{R}^3$:
    \begin{enumerate}
      \item Quin és el pla generat per la base $\{\overrightarrow{e_1},\overrightarrow{e_3}\}$?
      \item Com és el subespai generat per la base $\{\overrightarrow{e_2}\}$?
    \end{enumerate}
  \end{exercise}

  \begin{exercise}{}{}
    L'equació $2x-5y+7z=0$ defineix un subespai vectorial a $\mathbb{R}^3$. Quina és la seva base? Què representa l'equació?
  \end{exercise}

  \begin{exercise}{}{}
    Quins d'aquests vectors formen una base d'$\mathbb{R}^2$: $(1,1),(1,2),(-1,2)$?
  \end{exercise}

\end{frame}
\section{Sistemes de coordenades}
%------------------------------------------------
\begin{frame}
  \frametitle{Sistemes de coordenades}
  \begin{definicio}
    Un {\bf sistema de coordenades} $[O;\{\overrightarrow{v_1},\overrightarrow{v_2}, \ldots , \overrightarrow{v_n}\}]$ del producte cartesià $\mathbb{R}^n$ està format per:
    \begin{itemize}
      \item un punt $O\in\mathbb{R}^n$, anomenat origen i
      \item una base $\{\overrightarrow{v_1},\overrightarrow{v_2}, \ldots , \overrightarrow{v_n}\}$ de l'espai vectorial $\mathbb{R}^n$.
    \end{itemize}
  \end{definicio}
  Si triem $O=(0,0,.\ldots,O)$ i la base canònica $\{\overrightarrow{e_1},\overrightarrow{e_2}, \ldots , \overrightarrow{e_n}\}$ s'anomenen {\bf coordenades cartesianes}

  Les coordenades d'un punt $A$ respecte el sistema de coordenades $[O;\{\overrightarrow{v_1},\overrightarrow{v_2}, \ldots , \overrightarrow{v_n}\}]$ són les components del vector $\overrightarrow{OA}$ en la base $\{\overrightarrow{v_1},\overrightarrow{v_2}, \ldots , \overrightarrow{v_n}\}$.
\end{frame}
%------------------------------------------------%------------------------------------------------
\begin{frame}
%\frametitle{Discret vs contínu}
\begin{figure}
\includegraphics[width=0.4\linewidth]{FCTE}
\end{figure}
El dibuix representa el vector
\[
\vec{a}=a_x \vec{i} + a_y \vec{j} + a_z \vec{k}
\]
A $\mathbb{R}^3$ és habitual representar la base canònica com a $\{\vec{i},\vec{j},\vec{k}\}$.
\end{frame}
%------------------------------------------------%------------------------------------------------
\begin{frame}

  \begin{exercise}{ex:base}{}

    Considera el punt de coordenades cartesianes $P=(2,1)$ a $\mathbb{R}^2$:
    \begin{enumerate}
      \item Quines series les seves coordenades al sistema
      $[O;\{\overrightarrow{v_1}=(-1,1)_C,\overrightarrow{v_2}=(3,3)_C\}]$?
      \item Quines series les seves coordenades al sistema
      $[P;\{\overrightarrow{v_1}=(-1,1)_C,\overrightarrow{v_2}=(3,3)_C\}]$?
    \end{enumerate}
    \begin{figure}
    \includegraphics[width=0.5\linewidth]{FCTE}
    \end{figure}
  \end{exercise}
\end{frame}
%------------------------------------------------
\begin{frame}
%\frametitle{Discret vs contínu}
Resum de la solució a l'exercise:
\begin{columns}[c] % The "c" option specifies centered vertical alignment while the "t" option is used for top vertical alignment

\column{.45\textwidth} % Left column and width

\begin{figure}
\includegraphics[width=\linewidth]{FCTE}
\end{figure}
\column{.55\textwidth} % Right column and width
\begin{equation*}
(1,0)=\alpha(-1,1)+\beta(3,3)
\end{equation*}
amb resultat $\alpha=-\frac{1}{2}$ i $\beta=\frac{1}{6}$; i
\begin{equation*}
(0,1)=\gamma(-1,1)+\delta(3,3)
\end{equation*}
amb resultat $\gamma=\frac{1}{2}$ i $\delta=\frac{1}{6}$. i d'aquí:
\begin{eqnarray*}
(2,1)&=&2\cdot(1,0)+1\cdot(0,1)=\\
&=&-\frac{1}{2}\cdot(-1,1)+\frac{1}{2}\cdot(3,3)
\end{eqnarray*}
\end{columns}
\end{frame}
%------------------------------------------------%------------------------------------------------
\section{Trigonometria bàsica}
\begin{frame}
%\frametitle{Discret vs contínu}
\begin{definicio}
  Anomenem angle entre dues semirectes o segments amb un origen comú a la regió compresa entre ambdues semirectes o segments.
\end{definicio}
  \begin{figure}
    \includegraphics[width=0.3\linewidth]{FCTE}
    \includegraphics[width=0.6\linewidth]{FCTE}
  \end{figure}
\end{frame}
%-------------
\begin{frame}
  Mesurem els angles en graus ($\degree$) o en radiants. Un radiant és la rotació necessària per recórrer un arc de longitud igual al radi de la circumferència. Per tant, com que l'angle complet equvaldria a tota la circumferència és fàcil veure que $360\degree \equiv 2 \pi rad$.

  Per jugar amb aquest concepte proveu d'anar a \url{https://www.geogebra.org/m/kzc7rbNC}.
\end{frame}
%------------------------------------------------%------------------------------------------------
\begin{frame}
  \begin{exercise}{}
    Mesura l'angle formen els següents parells de vectors:
    \begin{itemize}
      \item  $(2,0)$ i $(0,5)$
      \item $(2,0)$ i $(0,-5)$
      \item $(2,0)$ i $(5,-5)$
    \end{itemize}
  \end{exercise}

  A $\mathbb{R}^3$, l'angle entre dos vectors es mesura damunt del pla que formen.

  \begin{exercise}{}{}
    Mesura l'angle formen els següents parells de vectors:
    \begin{itemize}
      \item  $(2,0,1)$ i $(0,5,0)$
      \item $(2,0,0)$ i $(0,-5,0)$
      \item $(2,0,0)$ i $(5,-5,5)$
    \end{itemize}
  \end{exercise}

\end{frame}
%------------------------------------------------%------------------------------------------------
\begin{frame}
  Està clar que en general no és tan trivial i cal fer ús de l'operació producte escalar, que definim com:
  \begin{definicio}
    S'anomena {\bf producte escalar} de dos vectors d'$\mathbb{R}^n$, definits per $\vec{u}=(u_1,u_2,\ldots,u_n)$ i $\vec{v}=(v_1,v_2,\ldots,v_n)$, al nombre real:
    \[
    \vec{u}\cdot \vec(v)= u_1\cdot v_1+u_1\cdot v_1+\cdots+u_1\cdot v_1.
    \]
  \end{definicio}
  Es pot demostrar que si els vectors formen un angle $\theta$, aleshores
  \[
  \vec{u}\cdot \vec(v)=||\vec{u}||\cdot ||\vec(v)|| \cdot \cos{\theta}
  \]
  i, per tant
  \[\cos{\theta}=\frac{\vec{u}\cdot \vec(v)}{||\vec{u}||\cdot ||\vec(v)||}
  \]
\end{frame}
%------------------------------------------------%------------------------------------------------
\begin{frame}
  \begin{exercise}{}{}
    Quin angle formen els vectors d'$\mathbb{R}^3$ $(1,1,1)$ i $(-1,1,-1)$?
  \end{exercise}

  \begin{exercise}{}
    Quan ha de valer $k$ perque els vectors $(1,k,1)$ i $(-1,1,-1)$ siguin paral$\cdot$lels? i perpendiculars?
  \end{exercise}

  \begin{exercise}{}{}
    Troba l'àrea del triangle delimitat pels vèrtex $(1,1),(4,5),(1,2)$.
  \end{exercise}
\end{frame}
%------------------------------------------------%-------
\begin{frame}
  \frametitle{Coordenades polars}
  Gràcies als angles podem descriure qualsevol posició al pla amb les anomenades {\bf coordenades polars}: $(r,\varphi)$, on
  \begin{itemize}
    \item $r$ és la distància del punt a l'origen de coordenades i
    \item l'angle $\varphi$ és el format pel vector i l'eix de les abcissses
  \end{itemize}
  Pots tafanejar aquesta URL: \url{https://www.geogebra.org/m/WTJq9yC9}
  \begin{figure}
    \includegraphics[width=0.35\linewidth]{FCTE}
  \end{figure}

\end{frame}
%------------------------------------------------%------------------------------------------------
\begin{frame}
  La conversió a coordenades cartesianes és:
  \begin{eqnarray*}
    x&=& r \cdot \cos{\varphi}\\
    y&=& r \cdot \sin{\varphi}
  \end{eqnarray*}
  i d'aquí podem treure la conversió contrària:
  \begin{eqnarray*}
    r&=& \sqrt{x^2+y^2}\\
    y&=& \arctan{\frac{y}{x}}
  \end{eqnarray*}
  on caldrà tenir en compte el quadrant alhora de calcular l'arctangent.

\end{frame}

%------------------------------------------------%------------------------------------------------
\begin{frame}
  Si enlloc d'$\mathbb{R}^2$ treballem sobre $\mathbb{R}^3$ ja no parlarem de coordenades polars sinó esfèriques, $(\rho,\varphi,\theta)$, on
  \begin{columns}[c]
    \column{0.5\textwidth}
  \begin{eqnarray*}
    x&=& \rho \cdot \sin{\varphi} \cdot \cos{\theta}\\
    y&=& \rho \cdot \sin{\varphi} \cdot sin{\theta}\\
    z&=& \rho \cdot \cos{\varphi}
  \end{eqnarray*}
    \column{0.5\textwidth}
  \begin{figure}
    \includegraphics[width=\linewidth]{FCTE}
  \end{figure}
\end{columns}

\end{frame}

%------------------------------------------------%------------------------------------------------
%------------------------------------------------%------------------------------------------------
%------------------------------------------------%------------------------------------------------
%------------------------------------------------%------------------------------------------------
%------------------------------------------------%------------------------------------------------
%------------------------------------------------%------------------------------------------------
%------------------------------------------------%------------------------------------------------
%------------------------------------------------%------------------------------------------------
%------------------------------------------------%------------------------------------------------
%------------------------------------------------%------------------------------------------------


%----------------------------------------------------------------------------------------

\end{document}

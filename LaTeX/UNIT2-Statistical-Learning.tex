%%%%%%%%%%%%%%%%%%%%%%%%%%%%%%%%%%%%%%%%%
% Beamer Presentation
% LaTeX Template
% Version 1.0 (10/11/12)
%
% This template has been downloaded from:
% http://www.LaTeXTemplates.com
%
% License:
% CC BY-NC-SA 3.0 (http://creativecommons.org/licenses/by-nc-sa/3.0/)
%
%%%%%%%%%%%%%%%%%%%%%%%%%%%%%%%%%%%%%%%%%

%----------------------------------------------------------------------------------------
%	PACKAGES AND THEMES
%----------------------------------------------------------------------------------------

\documentclass{beamer}

\include{commons_beamer.tex}
%----------------------------------------------------------------------------------------
%	 TITLE PAGE
%----------------------------------------------------------------------------------------

\title[Matrius]{Càlcul matricial} % The short title appears at the bottom of every slide, the full title is only on the title page

\author{Jordi Villà i Freixa} % Your name
\institute[FCTE] % Your institution as it will appear on the bottom of every slide, may be shorthand to save space
{
Universitat de Vic - Universitat Central de Catalunya \\
Grau en Multimèdia. Aplicacions i Videojocs\\ % Your institution for the title page
\medskip
\textit{jordi.villa@uvic.cat} % Your email address
}
%\date{\today} % Date, can be changed to a custom date
\date{07-19/10, 2022}
\logo{\includegraphics[width=.1\textwidth]{FCTE}}
\begin{document}

\begin{frame}
\titlepage % Print the title page as the first slide
\end{frame}

\begin{frame}
\frametitle{Índex} % Table of contents slide, comment this block out to remove it
\tableofcontents % Throughout your presentation, if you choose to use \section{} and \subsection{} commands, these will automatically be printed on this slide as an overview of your presentation
\end{frame}

%----------------------------------------------------------------------------------------
%	PRESENTATION SLIDES
%----------------------------------------------------------------------------------------

%------------------------------------------------
\section{Càlcul matricial} % Sections can be created in order to organize your presentation into discrete blocks, all sections and subsections are automatically printed in the table of contents as an overview of the talk
%------------------------------------------------

%\subsection{Subsection Example} % A subsection can be created just before a set of slides with a common theme to further break down your presentation into chunks
\begin{frame}
\frametitle{Referències}
El material d'aquestes presentacions està basat en anteriors presentacions i apunts d'altres professors \cite{jlgarcia,mcorbera,mcalle} de la UVic-UCC, pàgines web diverses (normalment enllaçades des del text), així com monografies \cite{vanverth,schaum,riley}.
\end{frame}

\begin{frame}
\frametitle{Definició de matriu}
Una {\bf matriu} és una caixa (plana) de números $a_{ij}\in \mathbb{R}$ ordenats per files i columnes:
\[
M=
\begin{pmatrix}
  a_{11} & a_{12} & \ldots & a_{1n}\\
  a_{21} & a_{22} & \ldots & a_{2n}\\
  \vdots & \vdots & \ddots & \vdots\\
  a_{m1} & a_{m2} & \cdots & a_{mn}
\end{pmatrix}
\]
Els subíndex indiquen la fila $i$ i la columna $j$ (en aquest ordre) a les qual pertany l'element $a_{ij}$ de la matriu. Es diu que una matriu com la descrita té dimensió o ordre $m \times n$ i anomenem $\mathscr{M}_{m\times n}$ el conjunt de totes les matrius de $m$ files i $n$ columnes.
\end{frame}




\begin{frame}
\begin{itemize}
  \item Una matriu és {\bf quadrada} si $m=n$
  \item La {\bf diagonal} d'una matriu és constituïda pels elements amb la mateixa fila que columna: $a_{ii}$
  \item Una matriu és diagonal si $a_{ij} = 0, \forall i\neq j$.
\end{itemize}
\end{frame}

\begin{frame}
\frametitle{SUMA de matrius}
Només es pot realitzar entre matrius del mateix ordre, i té les propietats, per a dues matrius $A,B \in \mathscr{M}_{m\times n}$:
\begin{description}
  \item[Commutativa] $A+B=B+A$
  \item[Associativa] $A+(B+C) = (A+B)+C$
  \item[Element neutre] $\exists 0 \in \mathscr{M}_{m\times n}: A+0=A$
  \item[Element oposat] $A+(-A)=0$
\end{description}
\begin{exercici}{}
  Posa exemples numèrics de cada propietat per a matrius d'ordre $2 \times 3$.
\end{exercici}
\end{frame}

\begin{frame}
\frametitle{PRODUCTE de matrius per un número real}
Per a dues matrius $A,B \in \mathscr{M}_{m\times n}$ i $\alpha,\beta \in \mathbb{R}$ es compleixen les següents propietats:
\begin{description}
  \item[Associativa] $\alpha \cdot (\beta \cdot A) = (\alpha \cdot \beta) \cdot A$
  \item[Element neutre] $A\cdot 1=A$
  \item[Distributiva respece la suma de matrius] $\alpha \cdot (A \pm B)=\alpha \cdot A \pm \alpha \cdot B$
  \item[Distributiva respecte la suma de reals] $(\alpha \pm \beta)\cdot A = \alpha \cdot A \pm \beta \cdot B$
  \item[Commutativa] $\alpha \cdot A = A \cdot \alpha$
\end{description}
\begin{exercici}{}
  Comprova que $\mathscr{M}_{m\times n}$ definida amb la suma de matrius i producte d'escalars és un espai vectorial.
\end{exercici}
\end{frame}

\begin{frame}
\frametitle{PRODUCTE de matrius}
Per a dues matrius $A \in \mathscr{M}_{m\times p}$ i $B \in \mathscr{M}_{p\times n}$, definim el producte $C=A\cdot B$ com una matriu $C \in \mathscr{M}_{m\times n}$ amb elements definit segons:
\[
c_{ij}=a_{i1}b_{1j}+a_{i2}b_{2j}+\cdots +a_{ip}b_{pj}
\]
Per exemple:
\[
\begin{pmatrix}
  -2 & -5\\
  1 & 2
\end{pmatrix}
\cdot
\begin{pmatrix}
  0 & 1 & 3\\
  1 & -1 & 1
\end{pmatrix}
=
\begin{pmatrix}
  -5 & 1 & -11\\
  2 & 0 & 5
\end{pmatrix}
\]

\begin{exercici}{}
  Comprova que no seria possible fer $
\begin{pmatrix}
  0 & 1 & 3\\
  1 & -1 & 1
\end{pmatrix}
\cdot
\begin{pmatrix}
  -2 & -5\\
  1 & 2
\end{pmatrix}$
\end{exercici}
\end{frame}


\begin{frame}
Propietats del producte de matrius (sepre que l'operació estigui definida segons s'explicava en l'anterior pàgina):
\begin{description}
  \item[No commutativa] (encara que siguin quadrades) $A\cdot B \neq B\cdot A$
  \item[Associativa] $(A\cdot B) \cdot C=A\cdot (B \cdot C)$
  \item[Distributiva] $A\cdot (B+C) = A\cdot B + A \cdot C$\\ $(B+C)\cdot A = B\cdot A + C \cdot A$ \\ $ \alpha\cdot (A \cdot B) = (\alpha \cdot A) \cdot B = A \cdot (\alpha \cdot B)$
  \item[Elements neutres] Si $A$ té ordre $m \times n$, aleshores $A \cdot I_n = A$ i $I_m \cdot A = A$, on $I_n$ i $I_m$ són les matrius quadrades diagonals amb 1's a la diagonal d'ordre $n$ i $m$, respectivament.
\end{description}
\begin{exercici}{}
  Posa exemples numèrics
\end{exercici}
\end{frame}

\begin{frame}
\begin{definicio}
  Per a matrius quadrades d'ordre $n$, es diu que $B$ és la matriu inversa de $A$ si $AB=BA=I_n$, escrivim $B=A^{-1}$ i diem que $A$ és invertible. Per exemple:
  \[
  \begin{pmatrix}
    1 & -2 & 0\\
    2 & 5 & 1\\
    -1 & 0 & 1
  \end{pmatrix}^{-1}
  =
  \frac{1}{11}
  \begin{pmatrix}
    5 & 2 & -2 \\
    -3 & 1 & -1\\
    5 & 2 & 9
  \end{pmatrix}
  \]
\end{definicio}
\begin{exercici}{}
  Comprova que $AA^{-1}=A^{-1}A=I_3$. Perquè creus que una matriu invertible ha de ser quadrada?
\end{exercici}
\end{frame}


\begin{frame}
  \frametitle{TRANSPOSICIÓ de matrius}
  Si $A$ és una matriu $m \times n$, aleshores $A^t$ és una matriu $n \times m$ resultant de canviar files per columnes.
  \[
  A= \begin{pmatrix}
  1 & -2 & 0\\
  2 & 5 & 1
\end{pmatrix}
\Rightarrow
A^t= \begin{pmatrix}
1 & 2\\
-2 & 5\\
0 & 1
\end{pmatrix}
  \]
  La transposició de matrius té les següents propietats:
  \begin{itemize}
    \item $(A+B)^t=A^t + B^t$
    \item $(r\cdot A)^t = r \cdot A^t$
    \item $(AB)^t= B^tA^t$
    \item $(A^t)^t =A$
    \item $(A^{-1})^t=(A^t)^{-1}$
  \end{itemize}
  \begin{exercici}{}
    Posa exemples numèrics
  \end{exercici}
\end{frame}

\section{Determinants}
\begin{frame}
  \frametitle{Determinants}
  Associem a cada matriu quadrada un nombre real anomenat determinant que es calcula segons:
  \begin{itemize}
    \item per a $n=1$, $A=(a)$; $det(a)=a$
    \item per a $n=2$, $A=\begin{pmatrix}a& b\\c & d\end{pmatrix}$; $det(A)= ad -bc$
    \item per a $n=3$, $A=\begin{pmatrix}a_{11}&a_{12}&a_{13}\\a_{21}&a_{22}&a_{23}\\a_{31}&a_{32}&a_{33}\end{pmatrix}$; $det(A)= +a_{11}a_{22}a_{33}+a_{12}a_{23}a_{31}+a_{21}a_{32}a_{13}-a_{22}a_{13}a_{31}+a_{33}a_{21}a_{12}+a_{11}a_{23}a_{32}$
  \end{itemize}
\end{frame}

\begin{frame}
  Per a $n \geq 4$, hem de descomposar el determinant d'ordre $n$ en termes d'ordre $n-1$ a partir del desenvolupament per a la fila $i=1$:\footnote{El resultat seria identic si enlloc de $i=1$ escollíssim qualsevol altra fila o columna}
  \begin{eqnarray}
  det(A)&=&a_{11}(-1)^{1+1}A_{11}+\\\nonumber
        &+& a_{12}(-1)^{1+2}A_{12}+\\\nonumber
        &+& a_{13}(-1)^{1+3}A_{13}+\\\nonumber
        &+& \cdots+\\\nonumber
        &+& a_{1n}(-1)^{1+n}A_{1n}
  \end{eqnarray}
  on $A_{ij}$ són els determinants de la matriu resultant d'extreure la fila $i$ i la columna $j$ (menors complementaris).
\end{frame}
%------------------------------------------------

\begin{frame}
  \begin{definicio}
    Una matriu $A\in \mathscr{M}_n$ és invertible si i només si $det(A) \neq 0$
  \end{definicio}
  En termes relatius a l'algebra, el determinant d'una matriu és zero si i només si existeix una fila/columna que és combinació lineal de les altres files/columnes.

  En aquest cas diem que les files/columnes són linealment dependents. Si el determinant dona diferent de zero, diem que són linealment independents.
\end{frame}
%------------------------------------------------%------------------------------------------------
\begin{frame}
  Algunes propietats dels determinants:
  \begin{itemize}
    \item $det(AB)=det(A)det(B)$
    \item $det(A)=det(A^t)$
    \item $det(A^{-1})=det(A)^{-1}= \frac{1}{det(A)}$
    \item $det(A+B) \neq det(A)+det(B)$ en general.
  \end{itemize}
\end{frame}

\section{Rang d'una matriu i dependència lineal}
%------------------------------------------------%------------------------------------------------
\begin{frame}
  \frametitle{Rang d'una matriu}

  \begin{definicio}
    El {\bf rang d'una matriu} és el major número de files o columnes linealment independents que conté la matriu.
  \end{definicio}
  Com podem trobar-lo?
  \begin{enumerate}
    \item Calculem "tots" els menors continguts a la matriu.
    \item Seleccionem el menor més gran i diferent de zero.
    \item El rang és l'ordre d'aquest menor.
  \end{enumerate}
\end{frame}
%------------------------------------------------%------------------------------------------------
\begin{frame}
  \begin{exercici}{}
    Troba el rang de les matrius
    \[A=
      \begin{pmatrix}
        1 & 0 & 1 & 1\\
        2 & 1 & 3 & 1\\
        3 & -1& 2 & 4
      \end{pmatrix}
    \]
    \[B=
      \begin{pmatrix}
        1 & 2 &-3 & 1 & 2\\
        2 & 4 &-4 & 6 & 10\\
        3 & 6 &-6 & 9 & 13
      \end{pmatrix}
    \]
  \end{exercici}
\end{frame}
%------------------------------------------------%------------------------------------------------
\begin{frame}
  \frametitle{Mètode del pivot de Gauss}
  Imaginem primer una matriu quadrada (té determinant). Cal tenir present que el determinant d'una matriu és idèntic al determinant de la matriu obtinguda després de fer combinació lineal de dues files/columnes. Això ens duu a usar aquesta propietat per reduir la complexitat del càlcul del determinant. Només haurem de tenir cura de no permutar l'ordre de les files/columnes; en aquest cas el determinant resultant canvia de signe.

  Segons això, el mètode del pivot de Gauss ens pot ajudar a calcular el rang. L'objectiu és fer zero tots els elements de fora la diagonal. En concret, podem:
  \begin{itemize}
    \item Multiplicar una fila/columna per un $\alpha \in \mathbb{R}: \alpha \neq 0$
    \item Sumar, a una fila, una altra fila multiplicada per un número
    \item Canviar l'ordre de dues files/columnes (el determinant canviarà de sgine)
  \end{itemize}
\end{frame}
%------------------------------------------------%------------------------------------------------
\begin{frame}
  Veiem-ho amb un exemple. Calculem el determinant de la matriu $A=\begin{pmatrix}-2&-1&2\\2&1&4\\-3&3&-1\end{pmatrix}$:

  \resizebox{\textwidth}{!}{
  \begin{tabular}{|c|c|c|c|c|}
    \hline\\
      Matriu      &   $B=\begin{pmatrix}-3&-1&2\\3&1&4\\0&3&-1\end{pmatrix}$
                  &   $C=\begin{pmatrix}-3&5&2\\3&13&4\\0&0&-1\end{pmatrix}$
                  &   $D=\begin{pmatrix}5&-3&2\\13&3&4\\0&0&-1\end{pmatrix}$
                  &   $E=\begin{pmatrix}18&-3&2\\0&3&4\\0&0&-1\end{pmatrix}$\\
    \hline
      Càlculs     &   $C_1+C_2 \rightarrow C_1$
                  &   $C_2+3*C_3 \rightarrow C_2$
                  &   $C_1 \leftrightarrow C_2$
                  &   $C_1+\frac{13}{3}C_2 \rightarrow C_1$\\
    \hline
      Det         &   $|A|=|B|$
                  &   $|B|=|C|$
                  &   $|D|=-|C|$
                  &   $|E|=|D|$\\
    \hline
  \end{tabular}
  }%closing resizebox

  Combinant aquestes igualtats obtenim: $|A|=-|E|=-18\cdot 3 \cdot (-1)=54$. En aquest cas, doncs, el rang de la matriu és 3.
\end{frame}
%------------------------------------------------%------------------------------------------------
\begin{frame}
  \frametitle{Independència lineal de vectors}
  Donada una base $C=\{ \overrightarrow{e_1},\overrightarrow{e_2},\ldots,\overrightarrow{e_m}\}$ i un conjunt de vectors $\{ \overrightarrow{u_1},\overrightarrow{u_2},\ldots,\overrightarrow{u_n}\}$ d'un espai vectorial fixat, amb components ($forall i = 1,\ldots,n$):
  \[
    \overrightarrow{u_i}=(a_{1i},a_{2i},\ldots,a_{mi})_C
  \]
  considerem la matriu de les components en forma de columnes:
  \[
    A=
      \begin{pmatrix}
        a_{11} & a_{12} & \cdots & a_{1n}\\
        a_{21} & a_{22} & \cdots & a_{2n}\\
        \vdots & \vdots & \ddots & \vdots\\
        a_{m1} & a_{m2} & \cdots & a_{mn}
      \end{pmatrix}
  \]
\end{frame}
%------------------------------------------------%------------------------------------------------
\begin{frame}
  \begin{block}{}$rang(A)=$número màxim de vectors linealment independents\end{block}

  \begin{exercici}{}
    Identifica els vectors linealment independents a la matriu
    $A=\begin{pmatrix}
      1&0&1&1\\
      2&1&3&1\\
      3&-1&2&4
    \end{pmatrix}$
  \end{exercici}

  \begin{exercici}{}
    Perquè tres vectors de $\mathbb{R}^2$ no formen mai una base?
  \end{exercici}
\end{frame}
%------------------------------------------------%------------------------------------------------
\section{Canvi de base}
\begin{frame}
  Donades dues bases $B=\{\overrightarrow{u_1},\ldots,\overrightarrow{u_n}\}$ i
  $D=\{\overrightarrow{v_1},\ldots,\overrightarrow{v_n}\}$ d'un espai vectoral determinat, suposem que coneixem l'expressió dels vectors de $B$ en termes dels de $D$, $\forall i = 1,\ldots,n$:
  \[
    \overrightarrow{u_1}= a_{1i}\cdot \overrightarrow{v_1}+
                          a_{2i}\cdot \overrightarrow{v_2}+\cdots+
                          a_{ni}\cdot \overrightarrow{v_n}
                        = (a_{1i},a_{2i},\ldots,a_{ni})_D
  \]
  \begin{block}{Matriu de canvi de base}
    S'anomena matriu de canvi de base de $B$ a $D$ a
    \[
      A_{B\overrightarrow D}=
      \begin{pmatrix}
        a_{11} & a_{12} & \cdots & a_{1n}\\
        a_{21} & a_{22} & \cdots & a_{2n}\\
        \vdots & \vdots & \ddots & \vdots\\
        a_{n1} & a_{n2} & \cdots & a_{nn}
      \end{pmatrix}
    \]
  \end{block}

\end{frame}
%------------------------------------------------%------------------------------------------------
\begin{frame}
  És l'eina que permet canviar les components d'un vector d'una base a l'altra:
  \[
    \overrightarrow{u_D}= A_{B \rightarrow D} \cdot \overrightarrow{u_B}
  \]
  on estem tractant un vector com una columna $n \times 1$.
  Exemple: siguin $B=\{\overrightarrow{u_1}=(1,1)_D,\overrightarrow{u_2}=(2,-1)_D\}$ i $D=\{\overrightarrow{v_1},\overrightarrow{v_2}\}$:
  \[
    A_{B \rightarrow D} = \begin{pmatrix}1&2\\1&-1\end{pmatrix}
  \]
  Si $\overrightarrow{u} = (3,5)_B$, aleshores
  \[
    \overrightarrow{u} = \begin{pmatrix}1&2\\1&-1\end{pmatrix}_{B \rightarrow D} \begin{pmatrix}3\\5\end{pmatrix}_B = \begin{pmatrix}13\\-2\end{pmatrix}_D
  \]
\end{frame}
%------------------------------------------------%------------------------------------------------
\begin{frame}
  \begin{exercici}{}
    Perquè una matriu de canvi de base sempre té determinant diferent de zero?
  \end{exercici}

  Com que aquest és el cas, la matriu té inversa, i:
  \[
    \overrightarrow{u_D}= A_{B \rightarrow D} \cdot \overrightarrow{u_B} \Leftrightarrow (A_{B \rightarrow D})^{-1} \cdot  \overrightarrow{u_D}=  \overrightarrow{u_B}
  \]
  És a dir, la matriu de canvi de base de $D$ a $B$ és la inversa de la matriu de canvi de base de $B$ a $D$:
  \[
    A_{D \rightarrow B} = (A_{B \rightarrow D})^{-1}
  \]
\end{frame}
%------------------------------------------------%------------------------------------------------
\begin{frame}
  Diem que dues bases $B$ i $D$ d'un espai vectorial tenen la mateixa orientació si el determinant del canvi de base és positiu. En cas contrari, les orientacions són oposades. Per exemple, a
  \begin{figure}
    \includegraphics[width=0.5\linewidth]{orient.png}
  \end{figure}
  \[
    A_{B \rightarrow D} = \begin{pmatrix}0&1&0\\1&0&0\\0&0&1\end{pmatrix}
  \]
  amb $|A|=-1<0$
  Per aprendre'n més: \url{https://mathinsight.org/determinant_linear_transformation}
\end{frame}


\begin{frame}
  Les matrius en el context geomètric ens faciliten enormement el càlcul:
  \begin{figure}
  \includegraphics[width=0.4\linewidth]{rotacioR2.png}
  \includegraphics[width=0.4\linewidth]{rotacioR3.jpg}
  \end{figure}
\end{frame}
%------------------------------------------------%------------------------------------------------

\begin{frame}
\frametitle{References}

\footnotesize{
\begin{thebibliography}{99} % Beamer does not support BibTeX so references must be inserted manually as below

\begin{columns}[t]
  \column{.45\textwidth}

\bibitem[VanVerth, 2015]{vanverth} James M. Van Verth, Lars M. Bishop (2015)
\newblock Essential Mathematics for Games and Interactive Applications
\newblock \emph{Elsevier}.

\bibitem[VanVerth, 2015]{vanverth} James M. Van Verth, Lars M. Bishop (2015)
\newblock Essential Mathematics for Games and Interactive Applications
\newblock \emph{Elsevier}.

\bibitem[Lipschutz, 2001]{schaum} Seymour Lipschutz, Marc Lipson (2001)
\newblock Schaum's outlines: Linear Algebra (3rd Ed)
\newblock \emph{McGraw Hill}.

\column{.45\textwidth}

\bibitem[Riley, 2002]{riley} K.F. Riley, M.P. Hobson, S.J. Bence (2002)
\newblock Mathematical Methods for Physics and Engineering (2nd Ed)
\newblock \emph{McGraw Hill}.

\bibitem[García]{jlgarcia} Josep Lluís García
\newblock Presentacions Matemàtiques Grau en Multimèdia, Aplicacions i Videojocs
\newblock \emph{UVic-UCC}.

\bibitem[Corbera]{mcorbera} Montserrat Corbera, Vladimir Zaiats
\newblock Apunts d'Àlgebra Lineal
\newblock \emph{UVic-UCC}.

\bibitem[Calle]{mcalle} Malu Calle
\newblock Apunts d'Àlgebra Lineal
\newblock \emph{UVic-UCC}.

\end{columns}
\end{thebibliography}

}
\end{frame}
%----------------------------------------------------------------------------------------

\end{document}
